%------------------------------------------------------------------------------
% CV in Latex
% Author : Charles Rambo
% Based off of: https://github.com/sb2nov/resume and Jake's Resume on Overleaf
% Most recently updated version may be found at https://github.com/fizixmastr 
% License : MIT
%------------------------------------------------------------------------------

\documentclass[A4,11pt]{article}
%\documentclass[letterpaper,11pt]{article} %For use in US
\usepackage{latexsym}
\usepackage[empty]{fullpage}
\usepackage{titlesec}
\usepackage{marvosym}
\usepackage[usenames,dvipsnames]{color}
\usepackage{verbatim}
\usepackage{enumitem}
\usepackage[hidelinks]{hyperref}
\usepackage[english]{babel}
\usepackage{tabularx}
\usepackage{tikz}
\usepackage[super]{nth}
\input{glyphtounicode}

\begin{comment}
I am by no means a professional when it comes to the CV's/resumes, I have
received various trainings on how to write a CV and resume from my high 
school, as well as the Austin College and University of Eastern Finland's
career counseling departments. As I intend to share my CV as a template, I 
feel that it is my responsibility to provide explanations of my work.
\end{comment}


%-----FONT OPTIONS-------------------------------------------------------------
\begin{comment}
The font of the document will impact not just how readable it is, but how it is
perceived. In the "The Craft of Scientific Writing" by Michael Alley, shares a
common fonts for publication as well as their use. I have chosen to use
Palatino for its legibility, some others are given below. There is far too much
about typography to discus here. Note: serif fonts have short projecting
strokes, sans-serif fonts are sans (without) these strokes.
\end{comment}


% serif
 \usepackage{palatino}
% \usepackage{times} %This is the default as well
% \usepackage{charter}

% sans-serif
% \usepackage{helvet}
% \usepackage[sfdefault]{noto-sans}
% \usepackage[default]{sourcesanspro}

%-----PAGE SETUP---------------------------------------------------------------

% Adjust margins
\addtolength{\oddsidemargin}{-1cm}
\addtolength{\evensidemargin}{-1cm}
\addtolength{\textwidth}{2cm}
\addtolength{\topmargin}{-1cm}
\addtolength{\textheight}{2cm}

% Margins for US Letter size
%\addtolength{\oddsidemargin}{-0.5in}
%\addtolength{\evensidemargin}{-0.5in}
%\addtolength{\textwidth}{1in}
%\addtolength{\topmargin}{-.5in}
%\addtolength{\textheight}{1.0in}

\urlstyle{same}

\raggedbottom
\raggedright
\setlength{\tabcolsep}{0cm}

% Sections formatting
\titleformat{\section}{
  \vspace{-4pt}\scshape\raggedright\large
}{}{0em}{}[\color{black}\titlerule \vspace{-5pt}]

% Ensure that .pdf is machine readable/ATS parsable
\pdfgentounicode=1

%-----CUSTOM COMMANDS FOR FORMATTING SECTIONS----------------------------------
\newcommand{\CVItem}[1]{
  \item\small{
    {#1 \vspace{-2pt}}
  }
}

\newcommand{\CVSubheading}[4]{
  \vspace{-2pt}\item
    \begin{tabular*}{0.97\textwidth}[t]{l@{\extracolsep{\fill}}r}
      \textbf{#1} & #2 \\
      \small#3 & \small #4 \\
    \end{tabular*}\vspace{-7pt}
}

\newcommand{\CVSubSubheading}[2]{
    \item
    \begin{tabular*}{0.97\textwidth}{l@{\extracolsep{\fill}}r}
      \text{\small#1} & \text{\small #2} \\
    \end{tabular*}\vspace{-7pt}
}

\newcommand{\CVSubItem}[1]{\CVItem{#1}\vspace{-4pt}}

\renewcommand\labelitemii{$\vcenter{\hbox{\tiny$\bullet$}}$}

\newcommand{\CVSubHeadingListStart}{\begin{itemize}[leftmargin=0.5cm, label={}]}
% \newcommand{\resumeSubHeadingListStart}{\begin{itemize}[leftmargin=0.15in, label={}]} % Uncomment for US
\newcommand{\CVSubHeadingListEnd}{\end{itemize}}
\newcommand{\CVItemListStart}{\begin{itemize}}
\newcommand{\CVItemListEnd}{\end{itemize}\vspace{-5pt}}

%\pagestyle{plain}

%------------------------------------------------------------------------------
% CV STARTS HERE  %
%------------------------------------------------------------------------------
\begin{document}

%-----HEADING------------------------------------------------------------------
\begin{comment}
In Europe it is common to include a picture of ones self in the CV. Select
which heading appropriate for the document you are creating.
\end{comment}

\begin{minipage}[c]{0.05\textwidth}
\-\
\end{minipage}
\begin{minipage}[c]{0.2\textwidth}
\begin{tikzpicture}
    \clip (0,0) circle (1.5cm);
    \node at (0,0) {\includegraphics[width = 2.5cm]{twhuang.jpg}}; 
    \draw (0,0) circle (1.5cm);
    % if necessary the picture may be moved by changing the at (coordinates)
    % width defines the 'zoom' of the picture
\end{tikzpicture}
\hfill\vline\hfill
\end{minipage}
\begin{minipage}[c]{0.6\textwidth}
    \textbf{\Huge \scshape{Tsung-Wei (TW) Huang}} \\ \vspace{2pt} 
    % \scshape sets small capital letters, remove if desired
    \small{} \\
    \href{mailto:tsung-wei.huang@utah.edu}{\underline{tsung-wei.huang@utah.edu}}\\
    % Be sure to use a professional *personal* email address
    % \href{https://www.linkedin.com/in/charles-rambo/}{\underline{linkedin.com/in/charles-rambo}} \\
    % you should adjust you linked in profile name to be professional and recognizable
    \href{https://tsung-wei-huang.github.io/}{\underline{https://tsung-wei-huang.github.io/}}
\end{minipage}

% Without picture
%\begin{center}
%    \textbf{\Huge \scshape Charles Rambo} \\ \vspace{1pt} %\scshape sets small capital letters, remove if desired
%    \small +1 123-456-7890 $|$ 
%    \href{mailto:you@provider.com}{\underline{you@provider.com}} $|$\\
%    % Be sure to use a professional *personal* email address
%    \href{https://linkedin.com/in/your-name-here}{\underline{linkedin.com/in/charles-rambo}} $|$
%    % you should adjust you linked in profile name to be professional and recognizable
%    \href{https://github.com/fizixmastr}{\underline{github.com/fizixmastr}}
%\end{center}



\begin{comment}
This CV was written for specifically for positions I was applying for in
academia, and then modified to be a template.

A standard CV is about two pages long where as a resume in the US is one page.
sections can be added and removed here with this in mind. In my experience, 
education, and applicable work experience and skills are the most import things
to include on a resume. For a CV the Europass CV suggests the categories: Work
Experience, Education and Training, Language Skills, Digital Skills,
Communication and Interpersonal Skills, Conferences and Seminars, Creative Works
Driver's License, Hobbies and Interests, Honors and Awards, Management and
Leadership Skills, Networks and Memberships, Organizational Skills, Projects,
Publications, Recommendations, Social and Political Activities, Volunteering.

Your goal is to convey a who, what , when, where, why for every item you share. 
The who is obviously you, but I believe the rest should be done in that order.
For example below. An employer cares most about the degree held and typically 
less about the institution or where it is located (This is still good 
information though). Whatever order you choose be consistent throughout.
\end{comment}

%-----POSITION----------------------------------------------------------------
\section{Appointment}
  \CVSubHeadingListStart
%    \CVSubheading % Example
%      {Degree Achieved}{Years of Study}
%      {Institution of Study}{Where it is located}
    \CVSubheading
      {{Assistant Professor, Department of Electrical and Computer Engineering}}{July 2019 -- Now}
      {University of Utah, Salt Lake City, Utah, USA}{}
    \CVSubheading
      {{Research Assistant Professor, Department of Electrical and Computer Engineering}}{2018 -- June 2019}
      {University of Illinois at Urbana-Champaign, IL, USA}{}
    %     \CVItemListStart
    %     \CVItem{Major GPA 3.5/4 (CGPA 3.41)}
    %     \CVItem {Ranking 9/46 (top 20\%)}
    %   \CVItemListEnd
  \CVSubHeadingListEnd

%-----EDUCATION----------------------------------------------------------------
\section{Education}
  \CVSubHeadingListStart
%    \CVSubheading % Example
%      {Degree Achieved}{Years of Study}
%      {Institution of Study}{Where it is located}
    \CVSubheading
      {{PhD, Department of Electrical and Computer Engineering }}{Aug. 2013 -- Dec. 2017}
      {University of Illinois at Urbana-Champaign, IL, USA}{}
    \CVSubheading
      {{MS, Department of Computer Science and Information Engineering}}{July 2010 -- July 2011}
      {National Cheng Kung University, Tainan, Taiwan}{}
    \CVSubheading
      {{BS, Department of Computer Science and Information Engineering}}{Sep 2006 -- June 2010}
      {National Cheng Kung University, Tainan, Taiwan}{}
    %     \CVItemListStart
    %     \CVItem{Major GPA 3.5/4 (CGPA 3.41)}
    %     \CVItem {Ranking 9/46 (top 20\%)}
    %   \CVItemListEnd
  \CVSubHeadingListEnd

%-----Research Interest--------------------------------------------
\section{Research Interest}
  \CVSubHeadingListStart
%    \CVSubheading % Example
%      {Work Presented}{When}
%      {Occasion}{}
    \CVSubheading % Example
     {High-performance Computing, Quantum Computing, Computer-aided Design}{}
     {Our software has been downloaded over 1.5M times being used by many organizations:}{}
     \CVItemListStart
       \CVItem{\textbf{Taskflow} (heterogeneous programming system): https://tsung-wei-huang.github.io/}
       \CVItem{\textbf{OpenTimer} (VLSI timing analysis tool): https://github.com/OpenTimer/OpenTimer}
       \CVItem{\textbf{RTLflow} (GPU-accelerated RTL simulator): https://github.com/dian-lun-lin/rtlflow}
       \CVItem{\textbf{SNIG} (sparse neural network inference): https://github.com/dian-lun-lin/SNIG}

       \CVItem{\textbf{DtCraft} (distributed cluster programming): https://github.com/twhuang-uiuc/DtCraft}
     \CVItemListEnd
  \CVSubHeadingListEnd  

% ------ Awards ------
\section{Awards}
 \begin{itemize}
 \itemsep-0.3em
    \item ACM SIGDA Meritorious Service Award, 2022
    \item Humboldt Research Fellowship Award, Alexander von Humboldt Foundation, 2022
    \item Faculty Early Career Development Program (CAREER) Award, NSF, 2022
    \item Best Paper Award for ``GPU-Accelerated Path-based Timing Analysis'', ACM TAU Workshop, 2021
    \item Champion of the IEEE/MIT/Amazon HPEC Large Sparse Neural Network Challenge, 2020
    \item \nth{2} Place (Taskflow), Open-source Software Competition, ACM Multimedia Conference, 2019
    \item ACM SIGDA Outstanding PhD Dissertation Award (thesis title: ``Distributed Timing Analysis''), 2019
    \item Best Tool Award (OpenTimer), Workshop on Open-source EDA Technology, 2018
    \item Best Open-source Software Award (DtCraft), ACM Multimedia Conference, 2018
    \item Best Poster Award for Open-source Parallel Programming Library (Taskflow), CPP Conference, 2018
    \item \nth{2} and \nth{1} Place, ACM/SIGDA CADathlon International Programming Contest, 2014 and 2017
    \item \nth{1}, \nth{2}, and \nth{1} Place, ACM TAU Timing Analysis Contest, 2014 through 2016
    \item Yi-Min Wang and Pi-Yu Chung Endowed Research Award, ECE Dept. UIUC, 2016
    \item Rambus Computer Engineering Fellowship, ECE Dept. UIUC, 2015—2016
    \item Study Abroad Scholarship, Ministry of Education, Taiwan, 2013—2014
    \item \nth{2} Place, ACM Student Research Competition Grand Final, ACM Annual Award Banquet, 2011
    \item Best Master’s Thesis Award, Taiwan Institute of Electrical and Electronic Engineering, 2011
    \item Best Master’s Thesis Award, IEEE Taiwan Tainan Section, 2011
    \item Best Master’s Thesis Award, Taiwan Institute of Information and Computing Machinery, 2011
    \item \nth{1} Place, Master’s Thesis Contest, Chinese Institute of Electrical Engineering, Taiwan, 2011
    \item Outstanding Graduate Recruiting Fellowship, National Cheng Kung University, 2010
    \item Outstanding Student Scholarship, Garmin Corporation, Taiwan, 2010
    \item \nth{1} Place, ACM/SIGDA Student Research Competition, Design Automation Conference, 2010
    \item \nth{3} Place, National Collegiate Cell-Based IC Design Contest, Ministry of Education, Taiwan, 2010
    \item Distinguished Engineering Student Fellowship, Chinese Institute of Engineers, Taiwan, 2009
    \item \nth{1} Place, National Collegiate Nano Device CAD Contest, Nano Device Laboratories, Taiwan, 2009
    \item \nth{3} Place, National Collegiate Programming Contest, Ministry of Education, Taiwan, 2009
    \item \nth{2} Place, National Collegiate IC/CAD Programming Contest, Ministry of Education, Taiwan, 2009
    \item \nth{2} Place, Presidential Award in CS Department, National Cheng Kung University, Taiwan, 2009

 \end{itemize}

% ---- Research Grants

\section{Research Grants}
  \CVSubHeadingListStart
%    \CVSubheading % Example
%      {Degree Achieved}{Years of Study}
%      {Institution of Study}{Where it is located}
    \CVSubheading
      {{Toward a Task-Parallel Programming Ecosystem for Modern Scientific Computing}}{}
      {PI, \$298K, NSF, TI-2229304}{Sep 2022 -- Aug 2023}
    \CVSubheading
      {{Developer Training Programs for Taskflow}}{}
      {PI, \$5K, NumFOCUS Small Development Grant}{Sep 2022 -- May 2023}
    \CVSubheading
      {{Transpiling Parallel Task Graph Programming Models for Scientific Software}}{}
      {PI, \$488K, NSF, OAC-2209957}{July 2022 -- July 2025}
    \CVSubheading
      {{Taskflow with Constrained Parallelis}}{}
      {PI, \$16K, NSF, CCF-2126672 (REU supplement)}{Aug 2022 -- Aug 2023}
    \CVSubheading
      {{Accelerating Static Timing Analysis with Intelligent Heterogeneous Parallelism}}{}
      {PI, \$500K, NSF, CCF-2144523 (CAREER)}{Jan 2022 -- Jan 2027}
    \CVSubheading
      {{GPU Acceleration for Static Timing Analysis}}{}
      {PI, \$10K (hardware donation), Nvidia Applied Research Acceleration Program}{Nov 2021}
    \CVSubheading
      {{A General-purpose Heterogeneous Task Graph Computing System for VLSI CAD}}{}
      {PI, \$403K, NSF, CCF-2126672}{Oct 2021 -- Oct 2024}
    \CVSubheading
      {{Standard GPU Algorithms with Task Graph Parallelism}}{}
      {PI, \$5K, NumFOCUS Small Development Grant}{May 2021 -- Feb 2022}
    \CVSubheading
      {{Taskflow-San: Sanitizing Erroneous Control Flows in Taskflow}}{}
      {PI, \$5K, NumFOCUS Small Development Grant}{May 2021 -- Feb 2022}
    \CVSubheading
      {{OpenTimer and DtCraft}}{}
      {PI, \$427K, DARPA, FA 8650-18-2-7843}{June 2018 -- July 2019}
    %     \CVItemListStart
    %     \CVItem{Major GPA 3.5/4 (CGPA 3.41)}
    %     \CVItem {Ranking 9/46 (top 20\%)}
    %   \CVItemListEnd
  \CVSubHeadingListEnd

% ------ PUBLICATION ------
\section{Conference Publication}
 \begin{enumerate}
 \itemsep-0.3em
    \item Dian-Lun Lin, Yanqing Zhang, Haoxing Ren, Shih-Hsin Wang, Brucek Khailany, and \underline{Tsung-Wei Huang}, ``GenFuzz: GPU-accelerated Hardware Fuzzing using Genetic Algorithm with Multiple Inputs,'' \textit{ACM/IEEE Design Automation Conference (DAC)}, San Francisco, CA, 2023
    \item \underline{Tsung-Wei Huang} ``qTask: Task-parallel Quantum Circuit Simulation with Incrementality,'' \textit{IEEE International Parallel and Distributed Processing Symposium (IPDPS)}, St. Petersburg, Florida, 2023 
    \item Guannan Guo, Martin D. F. Wong, and \underline{Tsung-Wei Huang}, ``Fast STA Graph Partitioning Framework for Multi-GPU Acceleration,'' \textit{IEEE/ACM Design, Automation and Test in Europe Conference (DATE)}, Antwerp, Belgium, 2023
    \item \underline{Tsung-Wei Huang} and Leslie Hwang, ``Task-Parallel Programming with Constrained Parallelism,'' \textit{IEEE High-performance Extreme Computing (HPEC)}, Waltham, MA, 2022
    \item \underline{Tsung-Wei Huang}, ``Enhancing the Performance Portability of Heterogeneous Circuit Analysis Programs,'' \textit{IEEE High-performance Extreme Computing (HPEC)}, Waltham, MA, 2022
    \item Dian-Lun Lin, Haoxing Ren, Yanqing Zhang, Brucek Khailany, and \underline{Tsung-Wei Huang}, ``From RTL to CUDA: A GPU Acceleration Flow for RTL Simulation with Batch Stimulus,'' \textit{ACM International Conference on Parallel Processing (ICPP)}, Bordeaux, France, 2022
    \item Cheng-Hsiang Chiu and \underline{Tsung-Wei Huang}, ``Composing Pipeline Parallelism using Control Taskflow Graph,'' \textit{ACM International Symposium on High-Performance Parallel and Distributed Computing (HPDC)}, Minneapolis, Minnesota, 2022
    \item Yu-Guan Chen, Chun-Yao Wang, \underline{Tsung-Wei Huang}, and Takashi Sato, ``Overview of 2022 CAD Contest at ICCAD,'' \textit{IEEE/ACM International Conference on Computer-aided Design (ICCAD)}, San Diego, CA, 2022
    \item Cheng-Hsiang Chiu and \underline{Tsung-Wei Huang}, ``Efficient Timing Propagation with Simultaneous Structural and Pipeline Parallelisms,'' \textit{ACM/IEEE Design Automation Conference (DAC)}, San Francisco, CA, 2022 
    \item \underline{Tsung-Wei Huang} and Yibo Lin, ``Concurrent CPU-GPU Task Programming using Modern C++,'' \textit{International Workshop on High-Level Parallel Programming Models and Supportive Environments (HIPS)}, France, 2022
    \item Kexing Zhou, Zizheng Guo, \underline{Tsung-Wei Huang}, and Yibo Lin, ``Efficient Critical Paths Search Algorithm using Mergeable Heap,'' \textit{IEEE/ACM Asia and South Pacific Design Automation Conference (ASPDAC)}, Taiwan, 2022
    \item Guannan Guo, \underline{Tsung-Wei Huang}, and Martin Wong, ``GPU-accelerated Path-based Timing Analysis,'' \textit{ACM/IEEE Design Automation Conference (DAC)}, CA, 2021
    \item Zizheng Guo, \underline{Tsung-Wei Huang}, and Yibo Lin, ``A Provably Good and Practically Efficient Common Path Pessimism Removal Algorithm for Large Designs,'' \textit{ACM/IEEE Design Automation Conference (DAC)}, CA, 2021
    \item McKay Mower, Luke Majors, and \underline{Tsung-Wei Huang}, ``Taskflow-San: Sanitizing Erroneous Control Flow in Taskflow Programs,'' \textit{IEEE Workshop on Extreme Scale Programming Models and Middleware (ESPM2)}, St. Louis, Missouri, 2021
    \item \underline{Tsung-Wei Huang}, ``TFProf: Profiling Large Taskflow Programs with Modern D3 and C++,'' \textit{IEEE International Workshop on Programming and Performance Visualization Tools (ProTools)}, St. Louis, Missouri, 2021
    \item Dian-Lun Lin and \underline{Tsung-Wei Huang}, ``Efficient GPU Computation using Task Graph Parallelism,'' \textit{European Conference on Parallel and Distributed Computing (Euro-Par)}, Portugal, 2021
    \item Yasin Zamani and \underline{Tsung-Wei Huang}, ``A High-Performance Heterogeneous Critical Path Analysis Framework,'' \textit{IEEE High-performance Extreme Computing (HPEC)}, Waltham, MA, 2021
    \item Cheng-Hsiang Chiu, Dian-Lun Lin and \underline{Tsung-Wei Huang}, ``An Experimental Study of SYCL Task Graph Parallelism for Large-Scale Machine Learning Workloads,'' \textit{International Workshop of Asynchronous Many-Task Systems for Exascale (AMTE)}, 2021
    \item Zizheng Guo, \underline{Tsung-Wei Huang}, and Yibo Lin, ``HeteroCPPR: Accelerating Common Path Pessimism Removal with Heterogeneous CPU-GPU Parallelism,'' \textit{IEEE/ACM International Conference on Computer-aided Design (ICCAD)}, Germany, 2021
    \item Guannan Guo, \underline{Tsung-Wei Huang}, Yibo Lin, and Martin D. F. Wong, ``GPU-accelerated Critical Path Generation with Path Constraints,'' \textit{IEEE/ACM International Conference on Computer-aided Design (ICCAD)}, Germany, 2021
    \item \underline{Tsung-Wei Huang}, Yu-Guan Chen, Chun-Yao Wang, and Takashi Sato, ``Overview of 2021 CAD Contest at ICCAD,'' \textit{IEEE/ACM International Conference on Computer-aided Design (ICCAD)}, Germany, 2021
    \item Kuan-Ming Lai, \underline{Tsung-Wei Huang}, Pei-Yu Lee, and Tsung-Yi Ho, ``ATM: A High Accuracy Extracted Timing Model for Hierarchical Timing Analysis,'' \textit{IEEE/ACM Asia and South Pacific Design Automation Conference (ASPDAC)}, Tokyo, Japan, 2021
    \item Chun-Xun Lin, \underline{Tsung-Wei Huang}, and Martin D. F. Wong, ``An Efficient Work-Stealing Scheduler for Task Dependency Graph,'' \textit{IEEE International Conference on Parallel and Distributed Systems (ICPADS)}, Hong Kong, 2020
    \item D.-L. Lin and \underline{Tsung-Wei Huang}, ``A Novel Inference Algorithm for Large Sparse Neural Network using Task Graph Parallelism,'' \textit{IEEE High-performance Extreme Computing (HPEC)}, Waltham, MA, 2020 (\textbf{Sparse Neural Network Graph Challenge Champion Award})
    \item Zizheng Guo, \underline{Tsung-Wei Huang}, and Yibo Lin, ``GPU-Accelerated Static Timing Analysis,'' \textit{IEEE/ACM International Conference on Computer-aided Design (ICCAD)}, San Diego, 2020 
    \item \underline{Tsung-Wei Huang}, ``A General-purpose Parallel and Heterogeneous Task Programming System for VLSI CAD,'' \textit{IEEE/ACM International Conference on Computer-aided Design (ICCAD)}, San Diego, 2020
    \item Ing-Chao Lin, Ulf Schlichtmann, \underline{Tsung-Wei Huang}, and Pao-Hun Lin, ``Overview of 2020 CAD Contest at ICCAD,'' \textit{IEEE/ACM International Conference on Computer-aided Design (ICCAD)}, San Diego, 2020
    \item G. Guo, \underline{Tsung-Wei Huang}, Chun-Xun Lin, and Martin D. F. Wong, ``An Efficient Critical Path Generation Algorithm Considering Extensive Path Constraints,'' \textit{ACM/IEEE Design Automation Conference (DAC)}, San Francisco, CA, 2020
    \item Chun-Xun Lin, \underline{Tsung-Wei Huang}, Guannan Guo, and Martin D. F. Wong, ``A Modern C++ Parallel Task Programming Library,'' \textit{ACM Multimedia Conference (MM)}, Nice, France, 2019 (\textbf{Second Prize of Open-Source Software Competition})
    \item Chun-Xun Lin, \underline{Tsung-Wei Huang}, Guannan Guo, and Martin D. F. Wong, ``An Efficient and Composable Parallel Programming Library,'' \textit{IEEE High-performance Extreme Computing (HPEC)}, Waltham, MA, 2019
    \item \underline{Tsung-Wei Huang}, Chun-Xun Lin, Guannan Guo, and Martin D. F. Wong, ``Cpp-Taskflow: Fast Task-based Parallel Programming using Modern C++,'' \textit{IEEE International Parallel and Distributed Processing Symposium (IPDPS)}, Rio De Janeiro, Brazil, 2019
    \item Kuan-Ming Lai, \underline{Tsung-Wei Huang}, and Tsung-Yi Ho, ``A General Cache Framework for Efficient Generation of Timing Critical Paths,'' \textit{ACM/IEEE Design Automation Conference (DAC)}, Las Vegas, NV, 2019
    \item \underline{Tsung-Wei Huang}, Chun-Xun Lin, Guannan Guo, and Martin D. F. Wong, ``Essential Building Blocks for Creating an Open-source EDA Project,'' \textit{ACM/IEEE Design Automation Conference (DAC)}, Las Vegas, NV, 2019
    \item \underline{Tsung-Wei Huang}, Chun-Xun Lin, and Martin D. F. Wong, ``Distributed Timing Analysis at Scale,'' \textit{ACM/IEEE Design Automation Conference (DAC)}, Las Vegas, NV, 2019
    \item \underline{Tsung-Wei Huang}, Chun-Xun Lin, Guannan Guo, and Martin D. F. Wong, ``A General-purpose Distributed Programming Systems using Data-parallel Streams,'' \textit{ACM Multimedia Conference (MM)}, Seoul, Korea, 2018 (\textbf{Best Open-Source Software Award}) 
    \item Chun-Xun Lin, \underline{Tsung-Wei Huang}, G. Guo, and Martin D. F. Wong, ``MtDetector: A High-performance Marine Traffic Detector at Stream Scale,'' \textit{ACM Distributed Event-based System Conference (DEBS)}, Hamilton, New Zealand, 2018
    \item Chun-Xun Lin, \underline{Tsung-Wei Huang}, T. Yu, and Martin D. F. Wong, ``A Distributed Power Grid Analysis Framework from Sequential Stream Graph,'' \textit{ACM Great Lakes Symposium (GLSVLSI)}, Chicago, IL, 2018
    \item Chun-Xun Lin, \underline{Tsung-Wei Huang}, and Martin D. F. Wong, ``Routing at Compile Time,'' \textit{IEEE International Symposium on Quality Electronic Design (ISQED)}, Santa Clara, CA, 2018
    \item \underline{Tsung-Wei Huang}, Chun-Xun Lin, and Martin D. F. Wong, ``DtCraft: A Distributed Execution Engine for Compute-intensive Applications,'' \textit{ACM/IEEE International Conference on Computer-aided Design (ICCAD)}, Irvine, CA, 2017
    \item Tin-Yin Lai, \underline{Tsung-Wei Huang}, and Martin D. F. Wong, ``An Effective and Accurate Macro-modeling Algorithm for Large Hierarchical Designs,'' \textit{ACM/IEEE Design Automation Conference (DAC)}, Austin, TX, 2017 (First Place of TAU Timing Analysis Contest)
    \item \underline{Tsung-Wei Huang}, Martin D. F. Wong, D. Sinha, K. Kalafala, and N. Venkateswaran, ``A Distributed Timing Analysis Framework for Large Designs,'' \textit{ACM/IEEE Design Automation Conference (DAC)}, Austin, TX, 2016
    \item \underline{Tsung-Wei Huang} and Martin D. F. Wong, ``OpenTimer: A High-performance Timing Analysis Tool,'' \textit{IEEE/ACM International Conference on Computer-aided Design (ICCAD)}, TX, 2015 (Second Place of TAU Timing Analysis Contest)
    \item \underline{Tsung-Wei Huang} and Martin D. F. Wong, ``On Fast Timing Closure: Speeding Up Incremental Path-Based Timing Analysis with MapReduce,'' \textit{IEEE/ACM International Workshop on System-level Interconnect Prediction (SLIP)}, CA, 2015
    \item \underline{Tsung-Wei Huang} and Martin D. F. Wong, ``Accelerated Path-Based Timing Analysis with MapReduce,'' \textit{ACM International Symposium on Physical Design (ISPD)}, Monterey, CA, 2015
    \item \underline{Tsung-Wei Huang}, P-C. Wu, and Martin D. F. Wong, ``Fast Path-Based Timing Analysis for CPPR,'' \textit{IEEE/ACM International Conference on Computer-aided Design (ICCAD)}, San Jose, CA, 2014 (First Place of TAU Timing Analysis Contest)
    \item \underline{Tsung-Wei Huang}, P.-C. Wu, and Martin D. F. Wong, ``UI-Timer: An Ultra-Fast Clock Network Pessimism Removal Algorithm,'' \textit{IEEE/ACM International Conference on Computer-aided Design (ICCAD)}, San Jose, CA, 2014 
    \item \underline{Tsung-Wei Huang}, P.-C. Wu, and Martin D. F. Wong, ``UI-Route: An Ultra-Fast Incremental Maze Routing Algorithm,'' \textit{IEEE/ACM International Workshop on System-level Interconnect Prediction (SLIP)}, San Francisco, CA, 2014
    \item S.-H. Yeh, J.-W. Chang, \underline{Tsung-Wei Huang}, and Tsung-Yi Ho, ``Voltage-Aware Chip-Level Design for Reliability-Driven Pin-Constrained EWOD Chips,'' \textit{IEEE/ACM International Conference on Computer-aided Design (ICCAD)}, San Jose, CA, 2012
    \item \underline{Tsung-Wei Huang}, J.-W. Chang, and Tsung-Yi Ho, ``Integrated Fluidic-Chip Co-Design Methodology for Digital Microfluidic Biochips,'' \textit{ACM International Symposium on Physical Design (ISPD)}, Napa, CA, 2012
    \item J.-W. Chang, \underline{Tsung-Wei Huang}, and Tsung-Yi Ho, ``An ILP-based Obstacle-Avoiding Routing Algorithm for Pin-Constrained EWOD Chips,'' \textit{IEEE/ACM Asia and South Pacific Design Automation Conference (ASPDAC)}, Sydney, Australia, 2012
    \item \underline{Tsung-Wei Huang}, Tsung-Yi Ho, and K. Chakrabarty, ``Reliability-Oriented Broadcast Electrode-Addressing for Pin-Constrained Digital Microfluidic Biochips,'' \textit{IEEE/ACM International Conference on Computer-aided Design (ICCAD)}, San Jose, CA, 2011
    \item \underline{Tsung-Wei Huang}, Yan-You Lin, J.-W. Chang, and Tsung-Yi Ho, ``Recent Research and Emerging Challenges in the Designs and Optimizations for Digital Microfluidic Biochips,'' \textit{IEEE System on Chip Conference (SOCC)}, 2011. 
    \item \underline{Tsung-Wei Huang}, Yan-You Lin, J.-W. Chang, and Tsung-Yi Ho, ``Chip-Level Design and Optimization for Digital Microfluidic Biochips,'' \textit{IEEE International Midwest Symposium on Circuits and Systems (MWSCAS)}, 2011. 
    \item P.-H. Yuh, C. C.-Y. Lin, \underline{Tsung-Wei Huang}, Tsung-Yi Ho, C.-L. Yang, and Y.-W. Chang, ``A SAT-Based Routing Algorithm for Cross-Referencing Biochips,'' \textit{IEEE/ACM International Workshop on System-level Interconnect Prediction (SLIP)}, San Diego, CA, June 2011.
    \item \underline{Tsung-Wei Huang}, H.-Y. Su, and Tsung-Yi Ho, ``Progressive Network-Flow Based Broadcast Addressing for Pin-Constrained Digital Microfluidic Biochips,'' \textit{ACM/IEEE Design Automation Conference (DAC)}, pp. 741—746, San Diego, CA, June 2011. 
    \item \underline{Tsung-Wei Huang}, S.-Y. Yeh, and Tsung-Yi Ho, ``A Network-Flow Based Pin-Count Aware Routing Algorithm for Broadcast Electrode-Addressing EWOD Chips,'' \textit{IEEE/ACM International Conference on Computer-aided Design (ICCAD)}, pp. 425-431, San Jose, CA, 2010. 
    \item \underline{Tsung-Wei Huang} and Tsung-Yi Ho, ``A Two-Stage Integer-Linear-Programming Based Droplet Routing Algorithm for Pin-Constrained Digital Microfluidic Biochips,'' \textit{ACM International Symposium on Physical Design (ISPD)}, pp. 201—208, San Francisco, CA, 2010. 
    \item \underline{Tsung-Wei Huang}, C.-H. Lin, and Tsung-Yi Ho, ``A Contamination-Aware Droplet Routing Algorithm for Digital Microfluidic Biochips,'' \textit{IEEE/ACM International Conference on Computer-aided Design (ICCAD)}, pp. 151—156, San Jose, CA, 2009. 
    \item \underline{Tsung-Wei Huang} and Tsung-Yi Ho, ``A Fast Routability- and Performance-Driven Droplet Routing Algorithm for Digital Microfluidic Biochips,'' \textit{IEEE International Conference on Computer Design (ICCD)}, pp. 445—450, Lake Tahoe, CA, 2009

 \end{enumerate}

\section{Journal Publication}
 \begin{enumerate}
 \itemsep-0.3em
  \item Dian-Lun Lin and \underline{Tsung-Wei Huang}, ``Accelerating Large Sparse Neural Network Inference using GPU Task Graph Parallelism,'' \textit{IEEE Transactions on Parallel and Distributed Systems (TPDS)}, vol. 33, no. 11, pp. 3041—3052, Nov 2022
  \item \underline{Tsung-Wei Huang}, Dian-Lun Lin, Chun-Xun Lin, and Yibo Lin, ``Taskflow: A Lightweight Parallel and Heterogeneous Task Graph Computing System,'' \textit{IEEE Transactions on Parallel and Distributed Systems (TPDS)}, vol. 33, no. 6, pp. 1303—1320, June 2022
  \item Zizheng Guo, Mingwei Yang, \underline{Tsung-Wei Huang}, and Yibo Lin, ``A Provably Good and Practically Efficient Algorithm for Common Path Pessimism Removal in Large Designs,'' \textit{IEEE Transactions on Computer-aided Design of Integrated Circuits and Systems (TCAD)}, vol. 41, no. 10, pp. 3466—3478, Oct. 2022
  \item Jia-Ruei Yu, Chun-Hsien Chen, \underline{Tsung-Wei Huang}, Jang-Jih Lu, Chia-Ru Chung, Ting-Wei Lin, Min-Hsien Wu, Yi-Ju Tseng, Hsin-Yao Wang, ``Energy Efficiency of Inference Algorithms for Medical Datasets: A Green AI study,'' \textit{Journal of Medical Internet Research (JMIR)}, to appear in 2022
  \item \underline{Tsung-Wei Huang}, Dian-Lun Lin, Yibo Lin, and Chun-Xun Lin, ``Taskflow: A General-purpose Parallel and Heterogeneous Task Programming System,'' \textit{IEEE Transactions on Computer-aided Design of Integrated Circuits and Systems (TCAD)}, vol. 41, no. 5, pp. 1448—1452, May 2022
  \item \underline{Tsung-Wei Huang}, Chun-Xun Lin, and Martin. D. F. Wong, ``OpenTimer v2: A Parallel Incremental Timing Analysis Engine,'' \textit{IEEE Design and Test (DAT)}, vol. 38, no. 2, pp. 62—68, April 2021
  \item \underline{Tsung-Wei Huang}, Yibo Lin, Chun-Xun Lin, G. Guo, and Martin. D. F. Wong, ``Cpp-Taskflow: A General-purpose Parallel Task Programming System at Scale,'' \textit{IEEE Transactions on Computer-aided Design of Integrated Circuits and Systems (TCAD)}, vol. 40, no. 8, pp. 1687—1700, Aug. 2021
  \item \underline{Tsung-Wei Huang}, G. Guo, Chun-Xun Lin, and Martin. D. F. Wong, ``OpenTimer v2: A New Parallel Incremental Timing Analysis Engine,'' \textit{IEEE Transactions on Computer-aided Design of Integrated Circuits and Systems (TCAD)}, vol. 40, no. 4, pp. 776—789, April, 2021
  \item \underline{Tsung-Wei Huang}, Chun-Xun Lin, and Martin D. F. Wong, ``DtCraft: A High-performance Distributed Execution Engine at Scale,'' \textit{IEEE Transactions on Computer-aided Design of Integrated Circuits and Systems (TCAD)}, vol. 38, no. 6, pp. 1070—1083, June 2018
  \item \underline{Tsung-Wei Huang} and Martin D. F. Wong, ``UI-Timer 1.0: An Ultra-Fast Path-Based Timing Analysis Algorithm for CPPR,'' \textit{IEEE Transactions on Computer-aided Design of Integrated Circuits and Systems (TCAD)}, vol. 35, no. 11, pp. 1862—1875, Nov. 2016
  \item S.-H. Yeh, J.-W. Chang, \underline{Tsung-Wei Huang}, S.-T. Yu, and Tsung-Yi Ho, ``Voltage-Aware Chip-Level Design for Reliability-Driven Pin-Constrained EWOD Chips,'' \textit{IEEE Transactions on Computer-aided Design of Integrated Circuits and Systems (TCAD)}, vol. 33, no.9, pp. 1302—1315, Sep. 2014. 
  \item J.-W. Chen, C.-L. Hsu, L.-C. Tsai, \underline{Tsung-Wei Huang}, and Tsung-Yi Ho, ``An ILP-Based Routing Algorithm for Pin-Constrained EWOD Chips with Obstacle Avoidance,'' \textit{IEEE Transactions on Computer-aided Design of Integrated Circuits and Systems (TCAD)}, vol. 32, no.11, pp. 1655—1667, Nov. 2013.
  \item Y.-H. Chen, C.-L. Hus, \underline{Tsung-Wei Huang}, and Tsung-Yi Ho, ``A Reliability-Oriented Placement Algorithm for Reconfigurable Digital Microfluidic Biochips using 3D Deferred Decision-Making Technique,'' \textit{IEEE Transactions on Computer-aided Design of Integrated Circuits and Systems (TCAD)}, vol. 32, no. 8, pp. 1151—1162, Aug. 2013.
  \item J.-W. Chang, S.-H. Yeh, \underline{Tsung-Wei Huang}, and Tsung-Yi Ho, ``Integrated Fluidic-Chip Co-Design Methodology for Digital Microfluidic Biochips,'' \textit{IEEE Transactions on Computer-aided Design of Integrated Circuits and Systems (TCAD)}, vol. 32, no 2, pp. 216—227, Feb. 2013.
  \item \underline{Tsung-Wei Huang}, S.-Y. Yeh, and Tsung-Yi Ho, ``A Network-Flow Based Pin-Count Aware Routing Algorithm for Broadcast-Addressing EWOD Chips,'' \textit{IEEE Transactions on Computer-aided Design of Integrated Circuits and Systems (TCAD)}, vol. 30, no. 12, pp. 1786—1799, Dec. 2011.
  \item \underline{Tsung-Wei Huang} and Tsung-Yi Ho, ``A Two-Stage Integer-Linear-Programming Based Droplet Routing Algorithm for Pin-Constrained Digital Microfluidic Biochips,'' \textit{IEEE Transactions on Computer-aided Design of Integrated Circuits and Systems (TCAD)}, vol. 30, no. 2, pp. 215—228, Feb. 2011. 
  \item \underline{Tsung-Wei Huang}, C.-H. Lin, and Tsung-Yi Ho, ``A Contamination-Aware Droplet Routing Algorithm for the Synthesis of Digital Microfluidic Biochips,'' \textit{IEEE Transactions on Computer-aided Design of Integrated Circuits and Systems (TCAD)}, vol. 29, no. 11, pp. 1682—1695, Nov. 2010. 

 \end{enumerate}

%----- Patent--------------------------------------------
\section{Patent}
  \CVSubHeadingListStart
%    \CVSubheading % Example
%      {Work Presented}{When}
%      {Occasion}{}
    \CVSubheading % Example
     {Incremental Common Path Pessimism Analysis}{USA-14/946043}
     {Tsung-Wei Huang, K. Kalafala, D. Sinha, and N. Venkateswaran}{}
    \CVSubheading % Example
     {Distributed Timing Analysis of a Partitioned Integrated Circuit Design}{USA-9916405B2}
     {Tsung-Wei Huang, K. Kalafala, D. Sinha, and N. Venkateswaran}{}
    %  \CVSubheading
    %  {}
  \CVSubHeadingListEnd  

% ------ PUBLICATION ------
\section{Talk}
 \begin{enumerate}
 \itemsep-0.3em
  \item ``Intelligent Heterogeneous Parallelism,'' ACCESS-CEDA Seminar Series at Hong Kong, Sep 2022
  \item ``Intelligent Heterogeneous Parallelism,'' CS Department, University of California at Merced, Sep 2022
  \item ``Programming System for Building High-performance CAD Applications,'' X Factory, Sep 2022
  \item ``A General-purpose Parallel and Heterogeneous Task Programming System,'' Xilinx, Aug 2022
  \item ``A GPU Acceleration Flow for RTL Simulation with Batch Stimulus,'' Invited Talk, IWLS, July 2022
  \item ``Intelligent Heterogeneous Computing,'' AMD Research, June 2022
  \item ``Intelligent Heterogeneous Computing,'' ECE Department, Johns Hopkins University, March 2022
  \item ``Intelligent Heterogeneous Computing,'' ECE Department, Stevens Institute of Technology, 2022
  \item ``Intelligent Heterogeneous Computing,'' ECE Department, University of Minnesota, Feb 2022
  \item ``Taskflow: A General-purpose Heterogeneous Task Programming System,'' IXPUG, 2021
  \item ``cudaFlow: A Modern C++ Programming Model for GPU Task Graph Parallelism,'' CppCon, 2021
  \item ``Taskflow: A General-purpose Heterogeneous Task Computing System,'' CUHK, Aug 2021
  \item ``HeteroTime: Accelerating Static Timing Analysis with GPUs,'' Nvidia Research, June 2021
  \item ``Taskflow: A Lightweight Heterogeneous Task Programming System,'' CPPNow, 2021
  \item ``GPU-Accelerated Static Timing Analysis and Beyond,'' GTC, April 2021
  \item ``Machine Learning-enabled System for EDA,'' VLSI-DAT, April 2021
  \item ``GPU-Accelerated Static Timing Analysis,'' UCSC EDA Seminar, Feb 2021 
  \item ``A General-purpose Heterogeneous Task Programming System,'' CIE/USA-GNYC, Oct 2020
  \item ``Taskflow: Parallel and Heterogeneous Task Programming in C++,'' C++ Meetup, Oct 2020
  \item ``Taskflow: A General-purpose Heterogeneous Task Programming System,'' CppIndia, Oct 2020
  \item ``Taskflow: A General-purpose Heterogeneous Task Programming System,'' MUC++, Oct 2020
  \item ``Programming Systems for Parallelizing VLSI CAD and Beyond,'' VLSI-DAT, April 2020
  \item ``A General-purpose Heterogeneous Task Programming System at Scale,'' ORNL, March 2020
  \item ``Growing Your Open-Source Projects,'' WOSET at IEEE/ACM ICCAD, November 2019
  \item ``Essential Building Blocks for Creating an Open-source EDA Project,'' IEEE/ACM DAC, June 2019
  \item ``Task-based Parallel Programming using Modern C++'', CSL Social Hour, Sep 2018
  \item ``Distributed Timing Analysis in 100 Lines of Code,'' VSD webinar, May 2018
  \item ``DtCraft: A High-performance Distributed Execution Engine at Scale,'' CSLSC, UIUC, 2018
  \item ``OpenTimer: An open-source high-performance timing analysis tool,'' ORCONF, Italy, 2016
  \item ``Distributed Timing Analysis: Framework and Systems,'' Cadence, Austin, June 2016
  \item ``OpenTimer: A High-performance Timing Analysis Tool,'' Invited Talk, ICCAD, 2015
  \item ``Fast Path-based Timing Analysis,'' Invited Talk, ICCAD, 2014

 \end{enumerate}

%-----WORK EXPERIENCE----------------------------------------------------------
\begin{comment}
try to briefly explain what you did and why it is relevant to the position you
are seeking
\end{comment}

\section{Work Experience}
  \CVSubHeadingListStart
    \CVSubheading
      {{Software Engineer}}{May 2017 -- Aug 2017}
      {High-performance Computing Group, Citadel, Chicago, IL}{}
    \CVSubheading
      {{Software Engineer}}{May 2015 -- Aug 2015}
      {Timing Analysis Group, IBM, NY}{}
    \CVSubheading
      {{Software Engineer}}{May 2014 -- Aug 2014}
      {Timing Analysis Group, Mentor Graphics, Fremont, CA}{}
  \CVSubHeadingListEnd
    

\section{Service}
  \CVSubHeadingListStart
    \CVSubheading
      {{Chair/Co-chair, ICCAD CAD Contests}}{2020 -- 2023}
      {Organized contests to engage students in solvoing cutting-edge CAD problems}{}
    \CVSubheading
      {{Publicity Chair, IWLS}}{2020}
      {Promoted participation in International Workshop on Logic Synthesis}{}
    \CVSubheading
      {{Chair/Co-chair, ACM SIGDA CADathlon International Programming Contest}}{2018 -- 2021}
      {Organized real-time contests to engage students in solvoing fundamental CAD problems}{}
    \CVSubheading
      {{Co-chair, ACM TAU Timing Analysis Contest}}{2018}
      {Organized contests to engage students in solving timing-related problems}{}
    \CVSubheading
      {{Chair, VSD Open Online EDA Conference}}{2018}
      {Organized the first online EDA conference with VSD at India}{}
     
    \CVSubheading
      {Program Committee}{}
      {Selected top-quality papers to organize conference programs}{}
      \CVItemListStart
        \CVItem{ACM/IEEE Design Automation Conference (DAC), 2022--2023}
        \CVItem{ACM TAU, 2020--2021}
        \CVItem{IEEE/ACM International Conference on Computer-aided Design (ICCAD), 2019--2022}
        \CVItem{IEEE/ACM Asia and South Pacific Design Automation Conference (ASPDAC), 2020--2021}
        \CVItem{IEEE International Conference on Computer Design (ICCD), 2020--2021}
        \CVItem{C++ Conference (CppCon), 2019--2021}
      \CVItemListEnd
    
    \CVSubheading
      {Editorship}{}
      {Managed peer-review processes and recommended what gets published}{}
      \CVItemListStart
        \CVItem{Guest editor, Special Issue of VLSI Integration, 2022}
      \CVItemListEnd
    
    \CVSubheading
      {Journal Reviewers}{}
      {Evaluated submitted papers and recommended acceptance/rejection}{}
      \CVItemListStart
        \CVItem{IEEE Transactions on Parallel and Distributed Computing Systems (TPDS)}
        \CVItem{IEEE Transactions on Computer-aided Design for Integrated Circuits and Systems (TCAD)}
        \CVItem{IEEE Transactions on Very Large-scale Integration (TVLSI)}
        \CVItem{IEEE Transactions on Circuits and Systems (TCAS)}
        \CVItem{IEEE Transactions on Big Data (TBD)}
        \CVItem{ACM Transaction son Design Automation of Electronic Systems (TODAES)}
        \CVItem{VLSI Integration Journal}
      \CVItemListEnd
    
    \CVSubheading
      {Departmental Committee at the University of Utah}{2019 -- Now}
      {Helped the ECE department enhance various research and teaching programs}{}
      \CVItemListStart
        \CVItem{Graduate Student and Admission Committee, 2021—-Now}
        \CVItem{University of Utah Asia Campus Committee, 2021—-Now}
        \CVItem{University of Utah Asia Campus Students Summer Visit Program, 2021}
        \CVItem{University of Utah Asia Campus faculty recruiting committee, 2021—-Now}
        \CVItem{Artificial Intelligence and Data-science faculty recruiting committee, 2020}
      \CVItemListEnd
  \CVSubHeadingListEnd
  
%-----Research Project--------------------------------------------
\begin{comment}
Again the title should have already been enough, but if it is necessary to add
descriptions maintain the consistency from prior sections
\end{comment}

%-----SE PROJECTS----------------------------------------------------
% \begin{comment}
% Ideally the title of the work should speak for what it is. However if you feel
% like you should explain more about why the project is applicable to this job,
% use item list as is shown in the work experience section.
% \end{comment}

% \section{Software Engineering Projects}
%   \CVSubHeadingListStart
% %    \CVSubheading
% %      {Title of Work}{When it was done}
% %      {Institution you worked with}{unused}
%     \CVSubheading
%       {{Iot Project on Smart Garden System built by ESP32} $|$ \emph{\small{C language}}}{Spring 2019}
%       {Xiamen University}{}
%     \CVSubheading
%       {{Big Data Analysis of Twitter} $|$ \emph{\small{Python}}}{Spring 2019}
%       {Xiamen University}{} 
%   \CVSubHeadingListEnd

%\section{Skills}
% \begin{enumerate}[leftmargin=0.5cm]
%    \small{\item{
%     \textbf{Languages}{: English (IELTS 6.5), Chinese (Native), Cantonese (Native)} \\
%     \textbf{Programming}{: Python (NumPy, SciPy, Matplotlib, Pandas), MATLAB, C \& C++, Java} \\
%     \textbf{Document Creation}{: Microsoft Office Suite, Latex, Markdown} \\
%    }}
% \end{enumerate}
    
%------------------------------------------------------------------------------
\end{document}
